\documentclass[11pt,a4paper]{book}
\usepackage[utf8]{inputenc}
\usepackage[english]{babel}
\usepackage{amsmath}
\usepackage{amsfonts}
\usepackage{amssymb}
\usepackage{graphicx}
\author{Ege Özkan}
\title{HUM203 \\ \large{Introduction to Social Antrophology Lecture Notes}}
\begin{document}
\newcommand{\unsure}{\textit{(?\textsuperscript{*})}}
\newcommand{\missed}{\textit{(!\textsuperscript{*})}}
\maketitle
\chapter{What is Antrophology}
Antropholoy is the study of humankind everywhere throughout time, steaming from Anthro (human) + logos (science of). In a world where most people live in multicultural and multiethnic states, importance of antrophology has increased.

Antrophology asks questions like:

\begin{enumerate}
\item Why are humans and cultures similar? (such as most cultures having marriage rituals)
\item Why are humans and cultures different? (such as differences between these rituals)
\item How does societies and cultures change, evolve and adopt.
\end{enumerate}

\section{Fields of Antropology}

Antrophology looks at the human experience in a broad manner, encompassing different systems of humanity with its psychological, cultural, social, biological and physical aspects. It is divided into fields such as:


\begin{itemize}
\item Cultural Antrophology
\item Archaeology
\item Linguistic Antrophology
\item Biological Antrophology
\end{itemize}

\subsection{Physical (Biological) Anthropology}

The systemic study of humans as biological organisms.

\subsubsection{Primotology}

Study of non-human primates. From lemurs to gorillas. They study these animals in the wild and in captivity. Primitalogists analyse how animals spend their time, how they behave.\\

Primotologists are concerned about extinction.

\subsubsection{Paleoanthropology}

Study of human evolution, including hominids.

\subsubsection{Contemporary Human Genetic Field}

They study the contemporory human genetics and biological makeup across diferent cultures and population groups.

\subsection{Archaeology}

The study of past human cultures through their material remains. It has two different subfields, pre-historic and historic archeologists. Divided by the time the writing started. The pre-historical archaeologists generaly define themselves with broad geographic zones.\\

Archeology may also be subdivided with respect to the specific field they research, such as industrial archeology or such as underwater archeology.

\subsection{Linguistic Anthropology}

The study of human languages, they look at their structure, history and relation to social and cultural contexts, as well its contemporary change. Language is the primary way humans communicate with each other, forming a basis of human cultures.

\subsection{Cultural Anthropology}

The study of customary patterns, thought and feelings, it focus on humans as culture-producing and culture-reproducing creatures. Cultural anthropologists spent time in the culture they research as field work, called \textit{Participant Observation}\unsure.

\chapter{Anthropological Approaches - October 26, 2020}

\section{Anthropological Perspectives}

Anthropology is differentiated from other fields by its perspectives.

\subsection{Holstic Perspective}

Holistic approach involves the analysis of biological enovrmental, psychological, economic, historical, social and cultural conditions of humanity. The holistic perspective, fundementally states that all aspects of culture can only be understood together. The holistic perspective says one must look for interconnections.

\subsection{Comperative Perspective}

Valid hypotheses and theories about humanity be tested wtih information from a wide range of cultures. This all-encompassing approach also guard against \textbf{culture-bound} theories.

\subsection{Relativistic Perspective}

The notation that one should not judge the behaviour of other peoples using their own culture.

\begin{description}
\item[Ethnocentrism] is the belief that the moral standards manners, attitudes, and so forth of ones own culture are superior to those of other cultures.
\item[Cultural relativism] means that no culture, taken as a whole, is inherently superior or inferior to any other.
\end{description}

Cultural relativism is differentiated to \textbf{methodological relativism} versus \textbf{moral/absolute relativism}. Moral relativism implies that there are no absolute, universal standards by which to evaluate actions in terms such as right and wrong or good and bad.\\

Methodological Relativsm is a methodological principle that refers to an outlook that is essential for maximum objectivity and understanding when studying a people whose way of life differs from their own.

\chapter{Culture - November 9, 2020}

\section{the Concept of Culture}
Concept of culture arose in the 19\textsuperscript{th} century. As a concept, Culture was defined as a complex whole that passed not biologically.\\

Culture does not need to be taught conciously. It is a learned and shared ideas and behaviour. It is made, artificial. Culture is the predominant feature differentiating human sfrom each other.\\

\begin{description}
\item[Enculturation] Transmission of the cultural knowldge to the next generation.
\item[Culture Shock] is the psychological disoriantation experienced when attempting to operate in a radically different cultural environment.
\end{description}

Complex societies tend to contain \textit{sub-cultural groups}. These subcultural groups are considered subsets of the wider culture, retaining certain features from the national culture, but differ in some other ways. These societies consisting of numerous subcultures are called \textit{pluralistic societies}.\\

Cultures differ in two predominant ways. \textit{the ways of thinking} and \textit{the ways of behaviour}. Thiking is their perception of the world, what is going on inside their head, while behaviour is the way the commonly act.

\begin{description}
\item[Cultural Knowledge] All the information the children learn and adults apply, this is socially learned during \textbf{enculturation}.
\end{description}

\section{Cultural Knowledge}

\begin{description}
\item[Norm] Certain standards of behaviour people tend to follow and other's judge their peers on. When somoeone does not follow norm repeatedly, they will recieve a negative reaction.
\item[Values] People's beliefs about the way of life that is desirable for themselves and their society. They affect the motiavations of people. They may originate from family, religion or numerous other sources.
\item[Symbols] An object or action with conventional meanings.
\item[Constructions] The way a culture divides the reality into categories. Different cultures may define food differently.
\item[Worldviews] The way people interpret events, reality, their own image.
\end{description}

Culture provides the knowledge to adapt to the natural enviorenment. It acts as the basis for human social life. It provides readilly established norms, values, etc. It provides mental concepts and acts as a lens on how one percieve the word.\\

\section{Language and Culture}

Linguistic Anthropologists investistigate the connection between language and culture. Human ability to speak started with the evolution of a gene called FOXP2.\\

Evolution of language allowed the domination of the homo family as it allowed a higher ability to adapt.\\

Culture is largelly considered hard to understand without understanding a culture's language first.\\

Cultural contexts affect language, it is called the \textit{Cultural Emphasis of Knowledge}. This concept explains why some familial words do not exist in some languages but exist in others.\\

Sapir-Whorf hypotesis state that the language of a culture influence the view of reality. [Sapir-Whorf hypotesis is considered defunct in modern times, but it did give us the Arrival (short story and movie), so I say it worths it.] It states that the \textit{linguistic difference is the reason for cultural difference.} [Which is, like, absurd, Aztecs didn't sacrifice people because chicken and human sounded the same in Aztec language.]

Communication and technological developments have lead to fundemental changes in which young people read, write and share information, as well as the structuring of their communication.\\

Dialects may be viewed in the cultural lens. For instance, a dialect predominently spoken by a socioeceonomically disadvantaged groups may be considered \textit{bad}. While the dominant dialect, sometimes promoted by the aparatuses of the state, will be viewed as \textit{desirable}. 

\chapter{The Development of Anthropological Thought - November 16, 2020}

In the fifth century B.C.E, the greek historian Herodotus wrote about the peoples of Persia, northern Africa and nearby regions.\\

13\textsuperscript{th}, the Venetian trader Marco Polo reached China, [Whom was ruled by the Mongol Yui dynesty at the time.], his book depicting his adventures in China made his book popular among the European literary elite.\\

The 16\textsuperscript{th}century saw the rise of imperalism, and in the next 400 years, Spain, Portugal, Britain and France established colonies in newly discovered [, and previously unreachable] lands, causing an explosion of knowledge on the ways of living of distant peoples.\\

As the mid-19\textsuperscript{th} century rolled, new scientific questions started to rise, remains of Neandarthels and ancient tools were found, prompting questions of human orgins.\\

The existence of non-human hominids caused significant shockwaves to propagate throughout the scientific world, ousting Judeo-Christian influence over the fields of biology and anatomy.\\

This process did not occured overnight, important contributions to it included the publishing of the \textit{On the Origin of Species} by Charles Darwin, introducing the modern concept of \textit{Evolution}.\\

However, the biological concept of Darwinian Evolution started to find its way to the field of Anthropology, starting the formation of Cultural Darwinism.

\section{Unilineal Evolutionism}

The nineteenth-century theorotical orientation that held that all human ways of life pass through similar sequence of stages in their development, as the human cultures evolved they passed thorugh a series of stages.\\

E.B Tylor, an important contributor of Unilineal Evolutionary thinking theories that the religions also follow a simillary pattern, starting as Animist religions, proceeding to Polytheism and finally evolving to Monotheism.

\section{Historical Particularism}

Franz Boas believed that each culture is the \textit{unqiue} product of all the influences to which it was subjected in its past, making cross-cultural generalizations questionable.\\

Historical Particularism discredited the overly speculative schemes of Uniilinal Evolutionists, it insisted that fieldwork is the primary means of acquiring reliable information, it imparted the idea that cultural relativism as a methodological principle is esssential for the most accurate unuderstanding of another culture, it demonstrated and popularized the notion that cultural differences and biological differences have little to do with each other.

\end{document}